% notetaker    : dmitri
% proofreader  :

\subsection{2015-01-05 Monday}
\subsubsection{Review}
\THM{rectifiable}
\LET $\mathrm{S}\subset\mathbb{R}^n$ be bounded  \\
\THEN S is rectifiable \\
\IFF $\partial \mathrm{S}$ has measure zero, equivalently $\chi_s$ is integrable\\
\IDEA rectifiable is a property of the boundary of a set. If the boundary has measure zero then the set is rectifiable.\\

\DEFN{volume}
\LET $\mathrm{S}\subset\mathbb{R}^n$ be bounded and \textit{rectifiable}\\
\THEN define the volume of S as:\\
$V(S):=\int_{\mathrm{S}}1:=\int_{\mathrm{Q}}\chi_x$\\

\THM{partition of unity}
\LET $\mathrm{A}\subset\mathbb{R}^n$,
$\{\mathrm{U}_\alpha\}_{\alpha\in\mathcal{A}}$ be an open cover of A\\
\THEN $\exists$ a collection of $C^\infty$ functions $\{\psi_\beta\}_{\beta\in\mathcal{B}}$ s.t
\begin{myenumerate}
\item $\forall x\in\mathrm{A}\quad 0<\psi_\beta\leq1\quad\forall\beta\in\mathcal{B}$
\item $\forall x\in\mathrm{A}\quad\exists$ open neighbourhood V of $x$ such that:\\
  all but finitely many $\psi_\beta$ vanish on V (\textit{locally finite})
\item $\forall x\in\mathrm{A}\quad\sum_{\beta\in\mathcal{B}}\psi_\beta(x)\equiv 1$
\item $\forall x\in\mathrm{A}\quad\exists\;\alpha$ such that\\
  $\mathrm{supp}(\psi_\beta)\subset\mathrm{U}_\alpha$, ie
  $\{x|\psi_\beta(x)\neq0\}\subset\mathrm{U}_\alpha$
\end{myenumerate}
A collection of functions satisfying i, ii, iii is called a \textit{partition of unity}\\
It is \textit{subordinate} to the open cover $\{\mathrm{U}_\alpha\}_{\alpha\in\mathcal{A}}$
if it satisfies condition iv


\subsubsection{Open Sets}
\DEFN{extended integral}\label{defn:extended integral}
\LET $\mathrm{U}^{open}\subset\mathbb{R}^n$,
$f:\mathrm{U}\to\mathbb{R}$ is continuous\\
\IF $f\geq 0$\\
\THEN define then \textit{extended integral} of $f$ over U as:\\
$\int_{\mathrm{u}}f := \sup\{\int_{\mathrm{D}}f\:|\:
\mathrm{D}\subset\mathrm{U}\text{ where } \mathrm{D}\text{ is compact, rectifiable} \}$\\
\IF $f$ is arbitrary and
 $\int_{\mathrm{u}}f_+,\;\int_{\mathrm{u}}f_-$ exist\\
\THEN define then \textit{extended integral} of $f$ over U as:\\
$\int_{\mathrm{u}}f:=\int_{\mathrm{u}}f_+-\int_{\mathrm{u}}f_-$ where\\
$f_+(x)=\max\{f(x),0\}$ and $f_-(x)=\min\{-f(x), 0\}$\\

\THM{mnk 15.2}\label{thm:mnk 15.2}
\LET $\mathrm{U}^{open}\subset\mathbb{R}^n$,
$f:\mathrm{U}\to\mathbb{R}$ is continuous\\
Choose an \textit{exhaustion} of U by compact $K_i$ such that\\
$K_1\subset K_2^{\circ}\subset K_2\subset K_3^{\circ}\ldots$ and 
$\mathrm{U}=\cup_{i=1}^{\infty}\mathrm{K}_i$\\
$f$ has an \textit{extended integral}\\
\IFF the sequence $\int_{\mathrm{k}_i}|f_i|$ is bounded\\
\THEN $\int_{\mathrm{u}}f=\lim_{i\to\infty}\int_{\mathrm{k}_i}f$\\

\RMK
If $\mathrm{U}^{open}\subset\mathbb{R}^n$, then $\int_{\mathrm{u}}f$ refers to the
\textit{extended integral}\\

\THM{mnk 15.4}
\LET
 $\mathrm{U^{open}}$ is bounded,
$f:\mathrm{U}\to\mathbb{R}^n $ is bounded and continuous\\
\THEN
\begin{myenumerate}
\item the extended integral exists
\item if the ordinary integral exists, then they are equal
\end{myenumerate}

\THM{mnk 16.5}\label{thm:mnk 16.5}
\LET
$\mathrm{U}^{open}\subset\mathbb{R}^n$,
$f:\mathrm{U}\to\mathbb{R}$ is continuous,
$\{\psi_i\}$ be a partition of unity with compact support\\
\THEN $\int_{\mathrm{u}}f$ exists\\
\IFF $\sum_{i=1}^{\infty}\int_{\mathrm{u}}\psi_i|f|$ converges to a finite number\\
In this case $\int_{\mathrm{u}}f =\sum\limits_{i=1}^
\infty(\int_{\mathrm{u}}\psi_i f)$\\