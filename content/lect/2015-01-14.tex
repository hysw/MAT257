% notetaker    : dmitri
% proofreader  :

\subsection{2015-01-07 Wednesday}
Second test review
\subsubsection{Existence of the Integral}
\DEFN{integral}
\LET Q be a rectangle, $f:\mathrm{Q}\to\mathbb{R}$ be a bounded function, define:\\
$\overline{\int}_{\mathrm{Q}} f := \inf\{L(f, \mathrm{P})\}$ as the upper integral\\
$\underline{\int}_{\mathrm{Q}} f := \sup\{L(f, \mathrm{P})\}$ as the lower integral\\
\IF then the upper and lower sums agree\\
\THEN f is \textit{integrable} over Q\\

\THM{Riemann Condition}
\LET Q be a rectangle, $f:\mathrm{Q}\to\mathbb{R}$ be a bbd fn\\
\THEN upper and lower integral agree\\
\IFF given $\epsilon>0$ there is a partition P such that:\\
$U(f,P) - L(f,P)\leq\epsilon$\\
\IDEA necessary condition for the existence\\

\THM{mnk 11.2}
\LET $\mathrm{Q}\subset\mathbb{R}^n$, $f:\mathrm{Q}\to\mathbb{R}$ be a bounded function.\\
Define D to be the set of points for which f fails to be continuous\\
\THEN $\int_{Q} f$ exists\\
\IFF D has measure zero (f is \textit{almost continuous everywhere})\\

\THM{mnk 11.3}
\LET $\mathrm{Q}\subset\mathbb{R}^n$, 
$f:\mathrm{Q}\to\mathbb{R}$.
Assume $f$ is \textit{integrable}\\
\IF $f$ vanishes except on a set of measure zero\\
\THEN $\int_{\mathrm{Q}}f = 0$\\
\IF $f$ is non-negative and $\int_{\mathrm{Q}}f = 0$\\
\THEN $f=0$ almost everywhere

\subsubsection{Evaluation of the Integral}
\THM{Fundamental Theory of Calculus}
\LET $f:\mathrm{[a,b]}\to\mathbb{R}$ be continuous\\
\IF $F(x)=\int_a^xf(x)$ for $x\in[a,b]$\\
\THEN $D\int_a^xf=f(x)$\\
\IF $g$ is a function such that $g'(x)=f(x)\;\forall x$\\
\THEN $\int_a^x Dg = g(x)-g(a)$\\

\THM{Fubini's Theorem}
\LET $\mathrm{Q}=\mathrm{A}\times\mathrm{B}$,
where $\mathrm{A}\subset\mathbb{R}^k$ and
$\mathrm{B}\subset\mathbb{R}^n$.\\
$f:\mathrm{Q}\to\mathbb{R}$ be bdd,
write $f(x,y)$ for $x\in\mathrm{A}$ and 
$y\in\mathrm{B}$.\\
For each $x\in\mathrm{A}$ consider upper and lower integrals\\
$\underline{\int}_{y\in\mathrm{B}}f(x,y)$ and
$\overline{\int}_{y\in\mathrm{B}}f(x,y)$\\
\IF $f$ is integrable over Q\\
\THEN these two functions are integrable over $\mathrm{Q}$ and
$\int_{\mathrm{Q}}f = \int_{x\in\mathrm{A}}\underline{\int}_{y\in\mathrm{B}}f(x,y) =
\int_{x\in\mathrm{A}}\overline{\int}_{y\in\mathrm{B}}f(x,y)$


%\vfill
\subsubsection{Integral Over a Bounded Set}
\THM{mnk 13.5}
\LET $\mathrm{S}\subset\mathbb{R}^n$ be bounded, 
$f:\mathrm{S}\to\mathbb{R}$ be bounded and continuous function.
Define E to be the set of all points $x_0\in\partial S$ for which the condition
$\lim_{x\to x_0} f(x) = 0$ fails\\
\IF E has measure zero\\
\THEN $f$ is integrable over S\\
Converse also holds\\

\THM{mnk 13.6}
\LET $\mathrm{S}\subset\mathbb{R}^n$ be bounded, 
$f:\mathrm{S}\to\mathbb{R}$ be bounded and continuous function, 
and $\mathrm{A}=\mathrm{S}^{\circ}$.\\
\IF f is integrable over S\\
\THEN f is integrable over A and\\ 
$\int_{\mathrm{S}} f = \int_{\mathrm{A}}f$

\subsubsection{Rectifiable Sets}

\THM{mnk 14.1}
\LET $\mathrm{S}\subset\mathbb{R}^n$
\THEN S is \textit{rectifiable} \\
\IFF S bounded and $\partial S$ has measure zero \\

\THM{mnk 14.2}
Properties of rectifiable sets
\begin{myenumerate}
\item (Positivity). If S is rectifiable, $v(\mathrm{S})\geq 0$
\item (Monotonicity). If $\mathrm{S}_1$ and $\mathrm{S}_2$ are rectifiable with\\ 
$\mathrm{S}_1\subset\mathrm{S}_2$ then $v(\mathrm{S}_1)\leq v(\mathrm{S}_2)$
\item (Additivity). If $\mathrm{S}_1$ and $\mathrm{S}_2$ are rectifiable so are,\\
$\mathrm{S}_1\cup\mathrm{S}_2$ and $\mathrm{S}_1\cap\mathrm{S}_2$
\item Suppose S is rectifiable. Then $v(\mathrm{S})=0$ iff S has measure zero
\item If S is rectifiable, so is $\mathrm{S^{\circ}}$ and $v(\mathrm{S})=v(\mathrm{S}^{\circ})$
\item If S is rectifiable, and $f:\mathrm{S}\to\mathbb{R}$ is bounded continuous, 
then $f$ is integrable over S.
\end{myenumerate}


\DEFN{simple region}
\LET C be a compact and rectifiable set in $\mathbb{R}^{n-1}$,\\
$\phi,\psi:\mathrm{C}\to\mathbb{R}$ be continuous functions such that\\ 
$\phi(x)\leq\psi(x)\;\forall x\in\mathrm{C}$.\\
\THEN $\mathrm{S}\subset\mathbb{R}^n$ defined as \\
$\mathrm{S}:=\{(x,t)|x\in\mathrm{C}\text{ and } \phi(x)\leq t\leq\psi{x}\}$\\
is a simple region.
%\newpage

\THM{Fubini's Theorem for Simple Regions}
\LET 
$\mathrm{S}=\{(x,t)| x\in\mathrm{C}\text{ and } \phi(x)\leq t\leq\psi(s)\}$
be a simple region in $\mathbb{R}^n$ and let
$f:\mathrm{S}\to\mathbb{R}$ be a continuous function\\
\THEN $f$ is integrable over S and\\
$$\int_{\mathrm{S}}f = \int_{x\in\mathrm{c}}\int_{t=\phi(s)}^{t=\psi(x)}f(x,t)$$

\subsubsection{Extended Integrals}
Three definitions of the extended integral:
\begin{myenumerate}
\item\hyperref[defn:extended integral]{Extended Integral 1}
\item\hyperref[thm:mnk 15.2]{Extended Integral 2}
\item\hyperref[thm:mnk 16.5]{Extended Integral 3}
\end{myenumerate}

\THM{mnk 15.4}
\LET
 $\mathrm{U^{open}}$ is bounded,
$f:\mathrm{U}\to\mathbb{R}^n $ is bounded and continuous\\
\THEN
\begin{myenumerate}
\item the extended integral exists
\item if the ordinary integral exists, then they are equal
\end{myenumerate}

\subsubsection{Change of Variables}
\TODO

