\section{Tensors}

\secttoc

\subsection{Definition}

\begin{description}
\item[k-fold product $V^k$]
Let $V$ be a vector space. Define $V^k=\underbrace{V\times\ldots\times V}_{k}$, denote all k-tuple of vectors of $V$.
\item[multilinear function]
A function $f$ is multilinear if\\
$\begin{aligned}f(v_1, \ldots, v_i+v_i',\ldots, v_n)
                & = f(v_1, \ldots, v_i,\ldots, v_n)\\
                & + f(v_1, \ldots, v_i',\ldots, v_n)\end{aligned}$\\
and\\
$f(v_1, \ldots, av_i',\ldots, v_n)
  = af(v_1, \ldots, v_i,\ldots, v_n)$
\item[k-tensor] A multilinear function $f:V^k\to\bR$ is called a k-tensor on $V$. (also called tensor of order $k$)
\item[$\cL^k(V)$] The set of all k-tensors on $V$.
\item[$V^*\equiv \cL^1(V)$] \kw{dual space} of $V$.
\item[elementary tensor] $\phi_I$
\item[tensor product] $f\otimes g$
\item[exterior product] $\Lambda^k(W)$
\end{description}


\TODO inner product
\subsection{Elementary Tensor}
\MUNKRESREF{26.3}

Let $V$ be a vector space with basis $a_1, \ldots, a_n$.
Let $I=(i_1, \ldots, i_n)$ be a k-tuple of integers from set $\{1,\ldots, n\}$.
There is a unique k-tensor $\phi_I$ on $V$ such that for every tuple $J=(j_1, \ldots, j_n)$, such that
\[\phi(a_{j_1},\ldots,a_{j_n})=\begin{cases}1&\text{if }I=J\\0&\text{if }I\neq J\end{cases}\]
The tensors $\phi_I$ form a basis for $\cL^k(V)$.

The tensors $\phi_I$ are called \kw{elementary k-tensors} on $V$ corrisponding to the basis $a_1, \ldots, a_n$ for $V$.

\subsection{Tensor Product}
Let $f\in\cL^k(V), g\in\cL^l(V)$,
define $(f\otimes g)\in\cL^{k+l}(V)$ by
\[(f\otimes g)(v_1,\ldots,v_{k+l})=f(v_1,\ldots,v_k)\cdot g(v_{k+1},\ldots,v_{k+l})\]
$f\otimes g$ is multilinear and is called \kw{tensor product} of $f$ and $g$.

\subsubsection{Properties of tensor product}
\begin{description}
\item[Associativity] $f\otimes(g\otimes h)=(f\otimes g)\otimes h$
\item[Homogeneity] $cf\otimes g=c(f\otimes g)=f\otimes (cg)$
\item[Distributivity] Suppose $f,g\in \cL^k(V), h\in \cL^l(V)$ then
	\[(f+g)\otimes h=f\otimes h+g\otimes h\]
	\[h\otimes (f+g)=h\otimes f+h\otimes g\]
\end{description}
Given a basis $a_1,\ldots,a_n$ for $V$, the corrisponding elementary tensor $\phi_I$ satisfy the equation
\[\phi_I=\phi_{i_1}\otimes\cdots\otimes\phi_{i_k}\]

\subsection{Dual Transformation}
Suppose $T:V\to W$, define \kw{dual transformation} $T^*:\cL^k(W)\to\cL^k(V)$
by \[(T^*f)(v_1,\ldots,v_k)=f(T(v_1),\ldots,T(v_k))\]

\subsection{Symmetric Tensors}
A tensor $f\in\cL^k(V)$ is called symmetric if\\
$f(v_1, \ldots, v_i, v_{i+1},\ldots, v_n) = f(v_1, \ldots, v_{i+1}, v_i,\ldots, v_n)$

\subsection{Alternating Tensors}
A tensor $f\in\cL^k(V)$ is called alternating if\\
$f(v_1, \ldots, v_i, v_{i+1},\ldots, v_n) = -f(v_1, \ldots, v_{i+1}, v_i,\ldots, v_n)$

The set of all alternating k-tensor is denoted $\cA^k(V)$ (Munkres use the notation $\cA^k(V)$).

\subsubsection{Exterior Algebra}
\TODO

\subsection{Wedge Product}
$\mathrm{Alt}(f)=\frac{1}{k!}A(f)=\sum_\sigma(\mathrm{sgn} \sigma)\cdot f^\sigma$

\extranote{Note that $\mathrm{Alt}$ was used in Spivak and $A$ was used in Munkres.}

Define wedge product by $f\wedge g = \frac{1}{k!l!}A(f\otimes g)$

\extranote{In Spivak's book $f\wedge g = \frac{(k+l)!}{k!l!}\mathrm{Alt}(f\otimes g)$.}
\subsubsection{Properties of wedge product}


\begin{description}
\item[Associativity] $f\wedge(g\wedge h)=(f\wedge g)\wedge h$
\item[Homogeneity] $cf\wedge g=c(f\wedge g)=f\wedge (cg)$
\item[Distributivity] Suppose $f,g\in \cL^k(V), h\in \cL^l(V)$ then\\
	$(f+g)\wedge h=f\wedge h+g\wedge h$\\
	$h\wedge (f+g)=h\wedge f+h\wedge g$
\item[Anticommutativity] $f\in \cL^k(V), g\in \cL^l(V)$ then\\
	$f \wedge g=(-1)^{kl}g \wedge f$
\item[property 5 in Munkres] \TODO
\end{description}

\TODO $T^*(f\wedge g)=T^*f\wedge T^*g$

\subsection{Theorems}
\theorem{26.1} \TODO
\theorem{26.2} \TODO