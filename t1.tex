\section{Measure zero}

\begin{description}
\item[measure zero] \hfill \\
	Let $A\subseteq \mathbb{R}^n$. We say $A$ has \underline{measure zero}
	in $\mathbb{R}^n$ if for every $\epsilon>0$, there is a covering $Q_1, Q_2, \dots$ of $A$
	by countably many rectangles such that
	$\sum_{i=1}^\infty v(Q_i)<\epsilon$.
	\extranote{If this inequality holds, we often say that the \underline{total volume} of the rectangles
	$Q_1, Q_2, \dots$ is less than $\epsilon$.}
\item[oscillation] \hfill \\
	Given $a\in Q$ define $A_\delta=\{f(x)|x\in Q \wedge |x-a|<\delta\}$.
	Let $M_\delta(f) = \sup A_\delta$, and let $m_\delta(f) = \inf A_\delta$,
	define oscillation at $f$ by $\mathrm{osc}(f;a)=\inf_{\delta>0}[M_\delta(f)-m_\delta(f)]$.
\end{description}

\begin{itemize}
 \item $f$ is cts at $a$ iff $\mathrm{osc}(f;a)=0$
\end{itemize}

\subsection{}
\subsubsection{Theorem \refto{Munkers-11.1}}

\begin{enumerate}
  \item If $B \subseteq A$ and $A$ has measure zero in $\mathbb{R}^n$, the so does B.
  \item Let $A$ be the union of the collection of sets
        $A_1, A_2, \ldots$ If each $A_i$ has measure zero, so does $A$.
  \item A set $A$ has measure zero in $\mathbb{R}^n$ if and only i 
\end{enumerate}

\subsubsection{Theorem \refto{Munkers-11.2}}




\subsection{Taylor’s theorem}
\label{taylor}
Suppose $f:\mathbb{R}^n \to \mathbb{R}$ is of class $C^k$ on an open convex set $S$. If $a \in S$ and $a+h \in S$, then
\begin{align*} 
f(a+h) = \sum\limits_{|\alpha| \leq k} \frac{\partial^\alpha f(a)}{\alpha!}h^\alpha + R_{a,k}(h),
\end{align*} 
If $f$ is of class $C^{k+1}$ on $S$, for some $c \in (0, 1)$ we have
\begin{align*} 
 R_{a,k}(h) = \sum\limits_{|\alpha| = k+1} \frac{\partial^\alpha f(a+ch)}{\alpha!}h^\alpha
\end{align*} 