\section{Differentiation}

\subsection{Derivative}
\label{derivative}

\subsubsection{Differentiable}
$f$ is differentiable at $a$ if there is an $n$ by $m$ matrix $B$ such that
			\[\frac{f(a+h)-f(a)-B\cdot h}{|h|}\to 0
			\quad\textrm{as}\quad h \to 0\]
		The matrix B is unique.

\subsubsection{Directional derivative}
Given $u\in \mathbb{R}^m$ which $u\neq 0$ define
		\[f'(a;u)=\lim_{t\to 0} \frac{f(a+tu)-f(a)}{t}\]
		Provide the limit exists.

\subsubsection{Partial derivative}
Define the $j^\textrm{th}$ partial derivative of $f$ at a
		to be the directional derivative of $f$ at $a$ with
		respect to the vector $e_j$, provide derivative exists.
		\[D_jf(a)=\lim_{t\to 0}\frac{f(a+te_j)-f(a)}{t}\]

\subsubsection{Continuously Differentiable}
A function is $C^1$ if all of its partial derivatives are continous. A function is $C^r$ if all of its partial derivatives are $C^{r-1}$.

\subsection{Theorems}
\begin{description}
	\item[Munkers.5.1] \hfill \\
		If $f$ is differentiable at $a$ then all directional
		derivative of $f$ at $a$ exist and
		$f'(a;u)=Df(a)\cdot u$
		
	\item[Munkers.5.2] \hfill \\
		If $f$ is differentiable at $a$ then $f$ is continuous at a.
		
	\item[Munkers.5.3] \hfill \\
		If $f$ is differentiable at $a$
		then $Df(a)=[D_1f(a) \quad D_2f(a) \quad \cdots \quad D_mf(a)]$.
		
	\item[Munkers.5.4] \hfill
	\begin{enumerate}
		\renewcommand{\labelenumi}{\alph{enumi}.}
		\item
		$[f \textrm{ is differentiable at } a ]
			\Leftrightarrow
			\forall i [f_i \textrm{ is differentiable at } a]$.
			
		\item
		If $f$ is differentiable at $a$, then its derivative is the n by m
		matrix whose $i^\textrm{th}$ row is the derivative of the function $f_i$. $(Df(a))_{i} = Df_i(a)$
	\end{enumerate}
\end{description}

\subsection{Continuously Differentiable Functions}

\begin{description}
	\item[Munkers 6.1] \hfill \\
		If $f:[a,b] \to \mathbb{R}$ is continuous on $[a,b]$ and differentiable on $(a,b)$, then there exists $c \in (a,b)$ such that $f(b)-f(a) = f'(c)(b-a)$.

	\item[Munkers 7.3] \hfill \\
		Let $A$ be open in $\mathbb{R}^m$; let $f:A \to \mathbb{R}$ be differentiable on $A$. If $A$ contains the line segment with end points $a$ and $a+h$, then there is a point $c = a+th$ with $0 < t < 1$ of this line segment such that $f(a+h)-f(a) = (Df(c))h$.		
	\item[Munkers 6.2] \hfill \\
		Let A be open in $\mathbb{R}^m$.
		Suppose that the partial derivative
		$D_if_i(x)$ of the component function
		of $f$ exists at each point $x$ of A and are continuous on A.
		Then $f$ is differentiable at each point of A.
		
	\item[Munkers 6.3] \hfill \\
		Let $A$ be open in $\mathbb{R}^m$, let $f: A \to \mathbb{R}$
		be a function of class $\mathbf{C}^2$. Then for each $a\in A$:
		$D_kD_jf(a)=D_jD_kf(a)$.
\end{description}

\subsection{Inverse Function Theorem}
\label{thm_ivft}
Let $A$ be open in $\mathbb{R}^n$.
	Let $f:A\to\mathbb{R}^n$ be of class $C^r$.

\thmIF
	$\ Df(x)$ is invertible at $a\in A$.
	
\thmTHEN
	$\ $There exists a neighborhood of $a$ such that
\begin{itemize}
	\item $f|_U$ is injective AND $f(U)=V$ open in $\mathbb{R}^n$
	\item the inverse function is of class $C^r$
	\item $f^{-1}(y)=[f'(f^{-1}(y))]^{-1}$
\end{itemize}


\subsection{Implicit Function Theorem}
\label{thm_ipft}
\begin{description}
	\item[Munkers 9.1] \hfill \\
		Let $A$ be open in $\mathbb{R}^{k+n}$, $B$ be open in $\mathbb{R}^k$.
		
		Let $f:A\to \mathbb{R}^n$, $g:B\to \mathbb{R}^n$ be differentiable.
		
		Write $f$ in the form $f(x, y)$, for $x\in \mathbb{R}^k$ and $y\in \mathbb{R}^n$.
		
		\thmIF $\ f(x, g(x))=0$ AND $\frac{\partial f}{\partial y}$ is invertible
		
		\thmTHEN $\ Dg(x)=-\left[\frac{\partial f}{\partial y}(x,g(x))\right]^{-1}\cdot \frac{\partial f}{\partial x}(x,g(x))$
\end{description}

Suppose $f:A\to \mathbb{R}^n$ be of class $C^r$.

Write $f$ in the form $f(x, y)$, for $x\in \mathbb{R}^k$ and $y\in \mathbb{R}^n$.

\thmIF $\ (a, b)\in A$ AND $f(a, b)=0$ AND $\det \frac{\partial f}{\partial y}(a,b)\neq 0$

\thmTHEN $\ $There exists $B\in \mathbb{R}^k, a \in B$ and a unique $g:B\to \mathbb{R}^n$ such that
		$g(a)=b$ AND $\forall x\in B. f(x,g(x))=0$ AND $g$ is $C^r$