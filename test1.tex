\section{Topics for First Midterm}
\subsection{Basic topology of metric spaces}
\begin{description}
\item[norm] $p:V\to\bR$\hfill\\
	For all $a\in F$ and all $u,v\in V$
	\begin{itemize}
	\item $p(v) \geq 0 \wedge [p(v) = 0 \iff v=0]$ (separates points)
	\item $p(av)=|a|p(v)$ (absolute homogeneity)
	\item $p(u + v) \leq p(u) + p(v)$ (triangle inequality)
	\end{itemize}
\item[inner product] $\langle x,y\rangle : V\times V \to \bR$\hfill\\
	For all $x,y,z \in V$ and $c\in\bF$.
	\begin{itemize}
	\item $\langle x,y\rangle = \langle y,x\rangle$
	\item $\langle x+y,z\rangle = \langle x,z\rangle + \langle y,z\rangle$
	\item $\langle cx,y\rangle = c\langle x,y\rangle = \langle x,cy\rangle$
	\item $\langle x,x\rangle > 0$ if $x\neq 0$
	\end{itemize}
\item[distance functions]$d: X \times X \to \bR$ \hfill\\
	For all $x, y, z \in X$
	\begin{itemize}
	\item $d(x, y)\geq 0$
	\item $d(x, y) = 0 \iff x=y$
	\item $d(x, y) = d(y, x)$
	\item $d(x, z) \leq d(x, y) + d(y, z)$
	\end{itemize}
\item[open and closed subsets of metric spaces]\hfill\\
	open: every point is a interior point \newline
	closed: iff contains all its limit point
\item[closure]
	the set with all its limit points
\item[interior]
	union of all open set contained in A
\item[exterior]
	union of all open set disjoint from A
\item[boundary]
	points that are neither interior nor exterior
\item[limits] $f:A\subseteq X\to Y$\hfill\\
$f(x)\to y_0 \textrm{ as } x\to x_0$ if
$\forall \mathrm{open} V\ni y_0\exists \mathrm{open} U\ni x_0 
	[x\in U\cap A \wedge x\neq x_0 \to f(x)\in V]$
\item[continuity] f is cts at $x_0$ if $x_0$ is isolated point or $(\lim_{x\to x_0}f(x)) = f(x_0)$
\item[Cauchy sequences] A sequences $\langle x_i\rangle$ is Cauchy if\hfill\\
	$\forall \varepsilon \exists N [n,m>N\implies |x_m-x_n|<\varepsilon6]$
\item[completeness]
	A metric space $X$ is complete if every Cauchy sequences converge(to some point in $X$).
\item[compact sets]
	every open cover of $X$ has a finite subcover
\item[connected sets]
	$X$ cannot be divided into two disjoint nonempty closed/open/clopen sets.
\item[relatively open sets] p26
	$A$ is relatively open in $Y\subseteq X$ if $\exists \mathrm{open}U\subseteq X [A=U\cap Y]$
\end{description}

Note, finite intersections and arbitrary unions of open set are open set

Cauchy-Schwarz inequality; all norms on a finite-dimensional vector
space are equivalent; Bolzano Weierstrass theorem; Heine-Borel theorem; the con-
tinuous image of a compact set is compact; the continuous image of a connected
set is connected; intermediate value theorem; extreme value theorem. minima and
maxima of continuous functions on compact sets
\subsection{Differentiation}


\begin{description}
\item[\nameref{derivative}]\hfill
	\begin{itemize}
		\item definition of the derivative
		\item partial derivatives
		\item directional derivatives
	\end{itemize}
\item[chain rule]\hfill
	\begin{itemize}
	\item$(f\circ g) = (f' \circ g)\cdot g'$
	\item$\frac{d f}{d x} = \frac{d f}{d g}\cdot \frac{d g}{d x}$
	\end{itemize}
\item[\nameref{thm_mvt}] see note
\item[\nameref{thm_ivft}] see note
\item[\nameref{thm_ipft}] see note
\item[continuity and differentiability]\hfill
	\begin{itemize}
	\item differentiable implies continuity
	\item $C^1$ implies differentiable
	\item $C^2$ implies equality of mixed partial derivatives
	\end{itemize}
\item[Jacobian matrix]\hfill\\
	$
	Jf = \begin{bmatrix}
	  \frac{\partial f_1}{\partial x_1} & \cdots & \frac{\partial f_1}{\partial x_n}\\
	  \ldots & \ddots & \ldots \\
	  \frac{\partial f_m}{\partial x_1} & \cdots & \frac{\partial f_m}{\partial x_n}
	 \end{bmatrix}
	$
\item[continuously differentiable functions]TODO
\item[higher order derivatives]TODO
\item[gradient]TODO
\item[geometry of the Jacobian, the rows, the columns]TODO
\end{description}


\subsection{Max-min problems}

\begin{description}
\item[\nameref{multiindex}] see note
\item[\nameref{taylor}] see note
\item[basic facts about the gradient]
\item[critical points]
\item[the Hessian]
\item[quadratic forms]
\item[classification of critical points]
\item[max-min problems with constraints]
\item[Lagrange multipliers]
\end{description}



\subsection{Instructor's comment}
General Comments: Should you memorize proofs of theorems? It is very hard to
memorize all proofs of all theorems. In the long run, it is much more efficient, as well
as useful and interesting, to first try to understand the proofs, and internalize the
methods of proof, as well as possible; then to remember just an outline of the proof,
or some key idea; roughly speaking, the minimum you would need to allow yourself
to reconstruct the proof out of your base of general knowledge/understanding.
Remember: it is important to know not simply whether something is true, but why
it is true.
